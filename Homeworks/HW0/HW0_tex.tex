\documentclass[11pt]{article}
\usepackage[margin=0.2in]{geometry}
% NOTE: Defining collaborators is optional; to not list collaborators, comment out the
% line below. Maximum of two collaborators per problem set.
%\newcommand{\collaborators}{Vig Vigerton (\texttt{UNI}), Alice Bob (\texttt{UNI})}
% No collaborators on PS 0

\makeatletter
\def\input@path{{../}{../../}} % add as many parents as you need
\makeatother
\input{../pset_template.tex}

\begin{document}

\section*{Problem 5: Linear Program Duality}

Lets first group constants and variables
\begin{align*}
		v + 0.5x_1 - 0.2x_2 - 0.3x_3 \ge 2 \\
		v - 0.1x_1 + 0.1x_2 \ge 3 \\
		v - 0.2x_1 - 0.95x_2 + 0.95x_3 \ge 5
\end{align*}

Written in standard form, the coefficient vector $c$, coefficient matrix $A$, and constant vector $b$ are:
$$
c =
\begin{pmatrix} 1 \\ 0 \\ 0 \\ 0
\end{pmatrix}, \quad
A =
\begin{pmatrix}
		1 & 0.5 & -0.2 & -0.3 \\
		1 & -0.1 & 0.1 & 0 \\
		1 & -0.2 & -0.95 & 0.95
\end{pmatrix}, \quad
b =
\begin{pmatrix} 2 \\ 3 \\ 5
\end{pmatrix}
$$

\subsection*{Dual LP Formulation}
Here, the dual variable vector is $y =
\begin{pmatrix} y_1 \\ y_2 \\ y_3
\end{pmatrix}$.

Our goal is to maximize $b^T y = 2y_1 + 3y_2 + 5y_3$.

The constraints are found by transposing $A$ and multiplying by $y$, then setting it equal to $c$:
$$
A^T =
\begin{pmatrix}
		1 & 1 & 1 \\
		0.5 & -0.1 & -0.2 \\
		-0.2 & 0.1 & -0.95 \\
		-0.3 & 0 & 0.95
\end{pmatrix}
$$
The constraints are:
\begin{align*}
		y_1 + y_2 + y_3 &= 1 \\
		0.5y_1 - 0.1y_2 - 0.2y_3 &= 0 \\
		-0.2y_1 + 0.1y_2 - 0.95y_3 &= 0 \\
		-0.3y_1 + 0.95y_3 &= 0
\end{align*}

\textbf{Sign Constraints:}
Since the primal constraints are of the "$\ge$" type, the dual variables must be non-negative: $y_1, y_2, y_3 \ge 0$.

\textbf{Final Dual Formulation:}
The complete dual LP is:
\begin{align*}
		\text{Maximize} \quad & 2y_1 + 3y_2 + 5y_3 \\
		\text{s.t.} \quad & y_1 + y_2 + y_3 = 1 \\
		& 0.5y_1 - 0.1y_2 - 0.2y_3 = 0 \\
		& -0.2y_1 + 0.1y_2 - 0.95y_3 = 0 \\
		& -0.3y_1 + 0.95y_3 = 0 \\
		& y_1, y_2, y_3 \ge 0
\end{align*}

\end{document}
