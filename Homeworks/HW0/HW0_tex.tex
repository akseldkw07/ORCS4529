\documentclass[11pt]{article}
\usepackage[margin=0.2in]{geometry}
% NOTE: Defining collaborators is optional; to not list collaborators, comment out the
% line below. Maximum of two collaborators per problem set.
%\newcommand{\collaborators}{Vig Vigerton (\texttt{UNI}), Alice Bob (\texttt{UNI})}
% No collaborators on PS 0

\makeatletter
\def\input@path{{../}{../../}} % add as many parents as you need
\makeatother
% Copyright 2021 Paolo Adajar (padajar.com, paoloadajar@mit.edu)
%
% Permission is hereby granted, free of charge, to any person obtaining a copy of this
% software and associated documentation files (the "Software"), to deal in the Software
% without restriction, including without limitation the rights to use, copy, modify,
% merge, publish, distribute, sublicense, and/or sell copies of the Software, and to
% permit persons to whom the Software is furnished to do so, subject to the following conditions:
%
% The above copyright notice and this permission notice shall be included in all copies
% or substantial portions of the Software.
%
% THE SOFTWARE IS PROVIDED "AS IS", WITHOUT WARRANTY OF ANY KIND, EXPRESS OR IMPLIED,
% INCLUDING BUT NOT LIMITED TO THE WARRANTIES OF MERCHANTABILITY, FITNESS FOR A
% PARTICULAR PURPOSE AND NONINFRINGEMENT. IN NO EVENT SHALL THE AUTHORS OR COPYRIGHT
% HOLDERS BE LIABLE FOR ANY CLAIM, DAMAGES OR OTHER LIABILITY, WHETHER IN AN ACTION OF
% CONTRACT, TORT OR OTHERWISE, ARISING FROM, OUT OF OR IN CONNECTION WITH THE SOFTWARE OR
% THE USE OR OTHER DEALINGS IN THE SOFTWARE.

%%%%%%%%%%%%%%%%%%%%%%%%%%%%%%%%%%%%%%
%%%%% DO NOT MODIFY THIS FILE %%%%%%%%
%%%%%%%%%%%%%%%%%%%%%%%%%%%%%%%%%%%%%%

%%%%%%%%%%%%%%%%%%%%%%%%%%%%%%
%%%%% CLASS SPECIFICS %%%%%%%%
%%%%%%%%%%%%%%%%%%%%%%%%%%%%%%
\newcommand{\classnum}{ORCS 4529}
\newcommand{\subject}{Reinforcement Learning}
\newcommand{\instructors}{Shipra Agrawal}
\newcommand{\semester}{Fall 2025}

%%%%%%%%%%%%%%%%%%%%%%%%%%%%%%
%%%%% PACKAGE IMPORTS %%%%%%%%
%%%%%%%%%%%%%%%%%%%%%%%%%%%%%%
\usepackage{fullpage}
\usepackage{enumitem}
\usepackage{amsfonts, amssymb, amsmath,amsthm}
\usepackage{tikz}
\usepackage{hyperref}
\usepackage{ifthen}

\hypersetup{
		colorlinks=true,
		linkcolor=blue,
		filecolor=magenta,
		urlcolor=blue,
}

%%%%%%%%%%%%%%%%%%%%%%%%%%%%
%%%%% CUSTOM MACROS %%%%%%%%
%%%%%%%%%%%%%%%%%%%%%%%%%%%%
\usepackage{macros}

%%%%%%%%%%%%%%%%%%%%%%%%%
%%%%% FORMATTING %%%%%%%%
%%%%%%%%%%%%%%%%%%%%%%%%%
\setlength{\parindent}{0mm}
\setlength{\parskip}{2mm}

\setlist[enumerate]{label=({\alph*})}
\setlist[enumerate, 2]{label=({\roman*})}

\allowdisplaybreaks[1]

%%%%%%%%%%%%%%%%%%%%%
%%%%% HEADER %%%%%%%%
%%%%%%%%%%%%%%%%%%%%%
\newcommand{\psetheader}{
		\ifthenelse{\isundefined{\collaborators}}{
				\begin{center}
						{\setlength{\parindent}{0cm} \setlength{\parskip}{0mm}

								{\textbf{\subject} \hfill \name}

								\textbf{\assignment} \hfill \href{mailto:\email}{\tt \email}

								\classnum:~\semester \hfill \textbf{Due:}~\duedate~11:59 PM ET

						\hrulefill}
				\end{center}
		}{
				\begin{center}
						{\setlength{\parindent}{0cm} \setlength{\parskip}{0mm}

								{\textbf{\subject} \hfill \name\footnote{Collaborator(s): \collaborators}}

								\textbf{\assignment} \hfill \href{mailto:\email}{\tt \email}

								\classnum:~\semester \hfill \textbf{Due:}~\duedate~11:59 PM ET

						\hrulefill}
				\end{center}
		}
}

%%%%%%%%%%%%%%%%%%%%%%%%%%%%%%%%%%%
%%%%% THEOREM ENVIRONMENTS %%%%%%%%
%%%%%%%%%%%%%%%%%%%%%%%%%%%%%%%%%%%
\newtheorem{theorem}{Theorem}
\newtheorem{lemma}[theorem]{Lemma}
\newtheorem{example}[theorem]{Example}


\begin{document}

\section*{Problem 5: Linear Program Duality}

Lets first group constants and variables
\begin{align*}
		v + 0.5x_1 - 0.2x_2 - 0.3x_3 \ge 2 \\
		v - 0.1x_1 + 0.1x_2 \ge 3 \\
		v - 0.2x_1 - 0.95x_2 + 0.95x_3 \ge 5
\end{align*}

Written in standard form, the coefficient vector $c$, coefficient matrix $A$, and constant vector $b$ are:
$$
c =
\begin{pmatrix} 1 \\ 0 \\ 0 \\ 0
\end{pmatrix}, \quad
A =
\begin{pmatrix}
		1 & 0.5 & -0.2 & -0.3 \\
		1 & -0.1 & 0.1 & 0 \\
		1 & -0.2 & -0.95 & 0.95
\end{pmatrix}, \quad
b =
\begin{pmatrix} 2 \\ 3 \\ 5
\end{pmatrix}
$$

\subsection*{Dual LP Formulation}
Here, the dual variable vector is $y =
\begin{pmatrix} y_1 \\ y_2 \\ y_3
\end{pmatrix}$.

Our goal is to maximize $b^T y = 2y_1 + 3y_2 + 5y_3$.

The constraints are found by transposing $A$ and multiplying by $y$, then setting it equal to $c$:
$$
A^T =
\begin{pmatrix}
		1 & 1 & 1 \\
		0.5 & -0.1 & -0.2 \\
		-0.2 & 0.1 & -0.95 \\
		-0.3 & 0 & 0.95
\end{pmatrix}
$$
The constraints are:
\begin{align*}
		y_1 + y_2 + y_3 &= 1 \\
		0.5y_1 - 0.1y_2 - 0.2y_3 &= 0 \\
		-0.2y_1 + 0.1y_2 - 0.95y_3 &= 0 \\
		-0.3y_1 + 0.95y_3 &= 0
\end{align*}

\textbf{Sign Constraints:}
Since the primal constraints are of the "$\ge$" type, the dual variables must be non-negative: $y_1, y_2, y_3 \ge 0$.

\textbf{Final Dual Formulation:}
The complete dual LP is:
\begin{align*}
		\text{Maximize} \quad & 2y_1 + 3y_2 + 5y_3 \\
		\text{s.t.} \quad & y_1 + y_2 + y_3 = 1 \\
		& 0.5y_1 - 0.1y_2 - 0.2y_3 = 0 \\
		& -0.2y_1 + 0.1y_2 - 0.95y_3 = 0 \\
		& -0.3y_1 + 0.95y_3 = 0 \\
		& y_1, y_2, y_3 \ge 0
\end{align*}

\end{document}
